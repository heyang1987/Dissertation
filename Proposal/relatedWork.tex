% !TeX root = proposal.tex
\chapter{Related Work}
In this chapter, we discuss existing research on privacy-setting interfaces and on privacy prediction, and finally we discuss why we use a data-driven design to recommending privacy preference in IoT.

\section{Personalization in IoT Systems}
One of the key features of IoT environments is that they have a high potential for providing personalized services to their users~\cite{vallee2016personalization, etzion2014personalization, hemant2015internet}. For example, Russell et al.~\cite{russell2015personalization} use unobtrusive sensors and micro-controller to realize a human detection for further providing personalization in a scenario of a family making use of the IoT in their daily living. Henka et al.~\cite{henka2016personalizing} propose an approach to personalize services in (household) IoT using the Global Public Inclusive Infrastructure's~\cite{vanderheiden2011creating} preference set to describe an individual's needs and preferences, and then adapting a smart environment accordingly.

\section{Privacy in Personalized systems}
Researchers have shown that privacy can play a limiting role in users' adoption of personalized services~\cite{teltzrow_2004}.  For example, Awad and Krishnan~\cite{awad_2006} show that privacy concerns inhibit users' use of personalized services, and Sutanto et al.~\cite{sutanto_2013} demonstrated that privacy concerns can prevent people from using a potentially beneficial personalized application. Kobsa et al.~\cite{kobsa_2016} demonstrate that the personalization provider is an important determinant of users' privacy concerns.

Moreover, research has shown  users' willingness to provide personal information to personalized services depends on both the risks and benefits of disclosure~\cite{phelps_2000,ho_2006,hui_2006}, and researchers therefore claim that both the benefits and the risks meet a certain threshold~\cite{treiblmaier_2007}, or that they should be in balance~\cite{chellappa_2005}.

\section{Privacy in IoT}
The argument that using user-generated data for personalization can result in privacy concerns has also been made in IoT environments~\cite{worthy_trust_2016}. 
	One of the first examples in this regard was the work by, Sheng et al.~\cite{sheng_experimental_2008}, who showed that users of ``u-commerce'' services (IoT-driven mobile shopping) felt less inclined to use personalized (rather than non-personalized) u-commerce services, unless the benefits were overwhelming (i.e., providing help in an emergency).

In response, researchers have proposed frameworks with guidelines for evaluating the security and privacy of consumer IoT applications, devices, and platforms~\cite{perera_privacy-by-design_2016, loi_systematically_2017}. Most of these guidelines focus on minimizing data acquisition, storage, and collection sources. Along these guidelines, several researchers have proposed architectures that restrict unwanted access to users' data by IoT devices. For example, Davies et al. propose ``privacy mediators'' to the data distribution pipeline that would be responsible for data redaction and enforcement of privacy policies even before the data is released from the user's direct control~\cite{davies_privacy_2016}. Likewise, Jayraman et al.'s privacy preserving architecture aggregates requested data to preserve user privacy~\cite{jayaraman_privacy_2017}.

Other research has considered IoT privacy from the end-user perspective~\cite{feth_user-centered_2017}, both when it comes to research (e.g., Ur et al. investigated how privacy perceptions differ among teens and their parents in smart security systems installed in homes~\cite{ur_intruders_2014}) and design (e.g., Williams et al. highlight the importance of designing interfaces to manage privacy such that they are usable to the end users of IoT devices~\cite{williams2016perfect}, and Feth et al. investigated the creation of understandable and usable controls~\cite{feth_user-centered_2017}). The current paper follows this approach, by outlining a novel methodology for the development of usable and efficient privacy-setting interfaces and applying it to household IoT privacy management. 

\section{Existing privacy control schemes}
Smartphones give users control over their privacy settings in the form of prompts that ask whether the user allows or denies a certain app access to a certain type of information. Such prompts are problematic for IoT, because IoT devices are supposed to operate in the background. Moreover, as the penetration of IoT devices in our homes continues to increase, prompts would become a constant noise which users will soon start to ignore, like software EULAs~\cite{good2005spyware} or privacy policies~\cite{jensen2004privacy}.

Pejovic and Musolesi~\cite{Pejovic2014} presented the design and implementation of an efficient online learner that can serve as a basis for recognizing opportune moments for interruption. The design of the library is based on an in-depth study of human interruptibility. Comparatively, our work tries to find the most suitable privacy-setting profile for each user based on their privacy preference on different household IoT scenarios.

\section{Privacy-Setting Interfaces}
Beyond prompts, one can regulate privacy with global settings. The most basic privacy-setting interface is the traditional ``access control matrix'', which allows users to indicate which entity gets to access what type of information~\cite{sandhu1994access}. This approach can be further simplified by grouping recipients into relevant semantic categories, such as Google+'s \emph{circles}~\cite{watson12}. Taking a step further, Raber et al.~\cite{197908} proposed \emph{Privacy Wedges} to manipulate privacy settings. Privacy Wedges allow users to make privacy decisions using a combination of semantic categorization (the various wedges) and inter-personal distance (the position of a person on the wedge). Users can decide who gets to see various posts or personal information by ``coloring'' parts of each wedge. 

Privacy wedges have been tested on limited numbers of friends, and in the case of household IoT they are likely to be insufficient, due to the complexity of the decision space. To wit, IoT privacy decisions involve a large selection of devices, each with various sensors that collect data for a range of different purposes. This makes it complicated to design an interface that covers every possible setting~\cite{williams2016perfect}. A wedge-based interface will arguably not be able to succinctly represent such complexity, and therefore either be impossible, or still lead to a significant amount of information and choice overload. 

We propose a data-driven approach to solve this problem: statistical analysis informs the construction of a layered settings interface, while machine learning-based privacy prediction helps us find smart privacy profiles.


\section{Privacy Prediction}
Several researchers have proposed privacy prediction as a solution to the privacy settings complexity problem---an approach known as ``user-tailored privacy'' (UTP)~\cite{knijnenburg2017}. Systems that implement UTP first predict users' privacy preferences and behaviors based on their known characteristics. They then use these predictions to provide automatic default settings or suggestions in line with users' disclosure profiles, to educate users' about privacy features they are unaware of, to tailor the privacy-setting user interfaces to make it easier for users to engage with their preferred privacy management tools, or to selectively restrict the types of personalization a system is allowed engage in.

Most existing work in line with this approach has focused on providing automatic default settings. For example, Sadeh et al.~\cite{sadeh2009understanding} used a k-nearest neighbor algorithm and a random forest algorithm to predict users' privacy preferences in a location-sharing system, based on the type of recipient and the time and location of the request. They demonstrated that users had difficulties setting their privacy preferences, and that the applied machine learning techniques can help users to choose more accurate disclosure preferences. Similarly, Pallapa et al.~\cite{pallapa2014adaptive} present a system which can determine the required privacy level in new situations based on the history of interaction between users. Their system can efficiently deal with the rise of privacy concerns and help users in a pervasive system full of dynamic interactions.

Dong et al.~\cite{dong2016ppm} use a binary classification algorithms to give users personalized advice regarding their privacy decision-making practices on online social networks. They found that J48 decision trees provided the best results. Li and et al.~\cite{li2017cross} similarly use J48 to demonstrate that taking the user's cultural background into account when making privacy predictions improves the prediction accuracy. Our data stems from a culturally homogeneous population (U.S. Mechanical Turk workers), so cultural variables are outside the scope of our study. We do however follow these previous works in using J48 decision trees in our prediction approach.

We further extend this approach using \emph{clustering} to find several smart default policies (``profiles''). This is in line with Fang et al.~\cite{fang2010privacy}, who present an active learning algorithm that comes up with privacy profiles for users in real time. Since our approach is based on an existing dataset, our algorithm does not classify users in real time, but instead creates a static set of profiles `offline', from which users can subsequently choose. This avoids cold start problems, and does not rely on the availability of continuous real-time behaviors. This is beneficial for household IoT privacy settings, because users often specify their settings in these systems in a ``single shot'', leaving the settings interface alone afterwards.

Ravichandran et al.~\cite{ravichandran2009capturing} employ an approach similar to ours, using $k$-means clustering on users' contextualized location sharing decisions to come up with several default policies. They showed that a small number of policies could accurately reflect a large part of the location sharing preferences. We extend their approach to find the best profiles based on various novel clustering approaches, and take the additional step of designing user interfaces that incorporate the best solutions.

\section{Data-driven design}
The development of usable privacy interfaces usually depends on user studies with existing systems. But what if an interface needs to be developed for a system that does not yet exist? In our first case --- general IoT environment, the Intel control framework has yet to be implemented~\cite{chow2015hci}. Thus the method using user study is not possible in our IoT control scenario. However, existing datasets can provide valuable input for UI development. We therefore develop and employ a \emph{data-driven design} methodology, leveraging an existing dataset collected by Lee and Kobsa~\cite{lee2016understanding}, who asked users whether they would allow or deny IoT devices in their environment to collect information about them. For our second and the third cases, household IoT and fitness IoT environments, we conducted separated survey studies to collect sufficient user data for our research. In the next three chapters, we are going to discuss our completed work in terms of recommending privacy preference for these three type of IoT environments using data-driven approach.


