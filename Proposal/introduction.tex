% !TeX root = proposal.tex
\chapter{Introduction}

% Introduction of Definition of IoT
Every passing day, our electronic device is getting smarter. It is no longer surprising that our refrigerator knows what food is stored inside it and notify us that we need to buy groceries when we start our car trying to go back home from work. Under the moniker of `Internet of Things` (IoT), smart connected devices are revolutionizing our everyday life. These smart devices ranging from personal devices~\cite{bahirat_exploring_2018,he_data_nodate} (e.g., fitness trackers, smart speakers, smart home appliances) to devices deployed in public areas and ``smart cities'' (e.g., smart billboards, RFID trackers, CCTV cameras)~\cite{bahirat2018data,lee_privacy_2017,pappachan_towards_2017}, are intended to collect information directly related to the users, such as fitness/healthy information, or the environment of users, such as users' home. A main feature of these smart devices, is that they are connected to a larger network of devices via local communication protocols and/or the Internet to create powerful new applications that supports our day-to-day activities. 

% The increasing appearance and impact of IoT devices
IoT is not a new word to normal users nowadays. Samsung's smart-things, Phillips' Hue smart lighting, Google's Nest smart learning thermostat, and ADT smart home security, Smart watches and fitness trackers, such as the Apple, Android and Pebble watches, Fitbit, Garmin, Jawbone, and Misfit bands, are helping us record our steps, heartbeats, and calories burnt. IoT has already established a huge impact in our everyday lives. As forecast by Gartner~\cite{van_der_meulen_gartner_nodate}, a total number of 21 billion IoT devices will be in use by 2020. This means that IoT devices are about to dethrone smartphones as the largest category of connected devices by then.

% The rapid acceleration brings both opportunity and problems
The rapid accelerating of the IoT brings a wealth of opportunity as well as risks. However, a lot of research has been focusing on the data and technology needs of the IoT -- the sensors, data, and the storage, security, and analysis of the data. However, research to an important aspect of IoT adoption and usage--the humans interacting with those technologies, are lacking. The demand for reducing the complexity and the burden in controlling these devices is urgent. Hence, my dissertation proposal research focuses on simplifying the task of controlling IoT devices for users using a data-driven design. People is bad at making decisions. This is also true in IoT privacy-setting domain [\textbf{need reference}]. To solve this problem, I use statistical analysis and machine learning to analyze how IoT device users make decisions regarding the privacy settings of their devices. Based on the insights gained from this analysis, I design intelligent User Interfaces to reduce the complexity of the privacy-setting user interface.

